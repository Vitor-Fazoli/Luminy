\chapter{Revisão da Literatura}
\label{cap:02}

Este capítulo será organizado de forma a se iniciar com as formas mais primitivas e os modelos de iluminação do mais simples para o mais complexo, logo após mostrar as técnicas de sombreamento e por fim mostrar as ferramentas e trabalhos correlatos ao proposto neste

\section{Técnicas de segmentação de imagens}

O autor  \citeauthorandyear{7359305} mostra em seus estudos que existem diversas técnicas para a segmentação de imagem e isso está bem longe das inteligencias artificiais existirem, como nos dias atuais, esses métodos por outro lado surgiram de conceitos matemáticos.
Essas técnicas usufruem de cada \textit{pixel} na imagem para entender a relação um a um com os seus adjacentes, fazendo com que possa ser medida uma similaridade entre os \textit{pixels}, 
o que torna a segmentação de imagem mais do que apenas um conjunto de informações realocados em outros contextos.

Para demonstrar como essas técnicas funcionam segundo \citeauthorandyear{ZAITOUN2015797} é necessário sub-dividir elas em alguns categorias, a primeira bifurcação encontrada está no ponto onde a segmentação é baseada ou em camadas ou em blocos, que será mostrado em seguida:
\section{\textit{Layer-Based Segmentation}}

O termo apresentado significa que a segmentação é baseada em camada, ou seja, segundo \citeauthorandyear{ZAITOUN2015797}  essa técnica utiliza múltiplos detectores de objetos para criar máscaras de forma, além de analisar a aparência e a profundidade dos objetos. 
Isso envolve a segmentação tanto de classes quanto de instâncias, permitindo uma compreensão mais detalhada da cena. 

\section{\textit{Block-Based Segmentation}}

Os Métodos de Segmentação de Imagem Baseados em Blocos, segundo  \citeauthorandyear{ZAITOUN2015797} são categorizados em duas propriedades: descontinuidade e similaridade, em três grupos, além de que aqui o processo de divisão dos grupos são categorizados de algumas formas
como cor, continuidade, similaridade e até mesmo bordas. baseado nesse processo surgiram diversos sub categorias que iremos tratar nas próximas sub-sessões 

\subsection{Métodos Baseados em Bordas (Edge-Based Methods)}
Os métodos baseados em bordas são focados em detectar descontinuidades na intensidade da imagem, ou seja, identificar transições abruptas que geralmente representam as bordas entre diferentes objetos ou regiões.

\subsubsection{Detecção de Bordas Clássica}
Entre os métodos clássicos, podemos destacar:
\begin{itemize}
    \item \textbf{Detecção de Bordas de Roberts}: Um método simples e eficiente que utiliza operadores cruzados para calcular o gradiente espacial. Sua simplicidade o torna ideal para aplicações que exigem baixo custo computacional.
    \item \textbf{Detecção de Bordas de Prewitt}: Calcula a magnitude e a orientação das bordas usando uma máscara de convolução de 3x3. Embora seja mais robusto que o método de Roberts, ainda é suscetível a ruídos.
    \item \textbf{Detecção de Bordas de Sobel}: Aplica uma máscara 3x3 rotacionada em 90º, o que permite uma melhor suavização de ruídos enquanto calcula o gradiente das bordas. É amplamente utilizado devido à sua eficácia em detecção de bordas.
\end{itemize}

\subsubsection{Métodos Suaves de Detecção de Bordas}
Com o avanço da inteligência artificial e da computação evolutiva, surgiram métodos mais sofisticados:
\begin{itemize}
    \item \textbf{Baseado em Lógica Fuzzy}: Usa conjuntos fuzzy para detectar bordas, permitindo que cada pixel pertença a múltiplas regiões. Isso oferece uma maior flexibilidade em imagens com transições suaves.
    \item \textbf{Baseado em Algoritmos Genéticos}: Inspirado na teoria da evolução, utiliza processos de seleção, cruzamento e mutação para identificar as bordas de maneira eficiente, especialmente em padrões complexos.
    \item \textbf{Baseado em Redes Neurais}: Redes neurais artificiais são treinadas para aprender padrões de bordas, ajustando os pesos entre suas camadas. São particularmente eficazes em detecção de bordas em cenários com grande variabilidade de padrões.
\end{itemize}

\subsection{Métodos Baseados em Regiões (Region-Based Methods)}
Os métodos baseados em regiões, por outro lado, focam na similaridade entre os pixels, agrupando-os em regiões contínuas com características semelhantes, como intensidade ou textura.

\subsubsection{Crescimento de Regiões}
O crescimento de regiões é um dos métodos mais intuitivos. Ele começa com um pixel inicial (semente) e expande a região ao adicionar pixels vizinhos que possuem propriedades semelhantes. Esse processo continua até que todos os pixels sejam alocados a uma região.

\subsubsection{Clustering}
\begin{itemize}
    \item \textbf{K-means}: Um dos métodos de clustering mais conhecidos, que particiona a imagem em K grupos ou clusters. Os pixels são atribuídos ao cluster cuja média é mais próxima. É amplamente utilizado, mas pode falhar ao lidar com regiões de forma complexa.
    \item \textbf{Clustering Fuzzy}: Diferente do K-means, esse método permite que um pixel pertença a mais de um cluster, proporcionando mais flexibilidade em imagens com transições suaves entre regiões.
\end{itemize}

\subsubsection{Técnica de Divisão e Fusão}
Neste método, a imagem é inicialmente dividida em regiões menores e, em seguida, regiões adjacentes são fundidas com base em critérios de similaridade. Essa abordagem é ideal para evitar problemas de supersegmentação.

\subsubsection{Corte Normalizado}
Baseado em teoria dos grafos, o corte normalizado visa minimizar o número de cortes entre regiões, resultando em uma segmentação mais eficiente. Cada pixel é tratado como um vértice no grafo, e as arestas são ponderadas com base na similaridade entre os pixels. Esse método é amplamente utilizado em aplicações de segmentação médica.

\section{Inteligencia artificial}

\section{Segment Anything}



\section{Trabalhos Correlatos}

Nesta seção são apresentados os trabalhos correlatos ao proposto neste trabalho

\subsection{Segmentation by grouping junctions}

Escrito por \citeauthorandyear{hiroshi_ishikawa__1998} este trabalho nos apresenta uma abordagem inovadora para a segmentação de imagens em tons de cinza, focando em junções, que são pontos críticos onde as bordas se encontram. 
Este método enfatiza a importância dessas junções na definição de regiões significativas dentro de uma imagem, contrastando com métodos tradicionais que frequentemente dependem apenas de informações de intensidade ou gradiente.

