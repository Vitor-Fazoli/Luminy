\chapter{Análise dos Resultados preliminares}
\label{cap:04}
 Inicialmente como resultado foi obtido a inicial analise feita na plataforma de iluminação \textit{Relight} que nos traz diversos pontos importantes para a criação da ferramenta.

Foi diagnosticado diversas funções e possíveis formas de como elas podem ter sido elaboradas. como por exemplo a utilização de camadas da imagem e a somatório de cores para criar a intensidade de luz sobre o que é apresentado.
\begin{comment}
Relatar os resultados obtidos a partir dos experimentos e dos estudos realizados. 


\section{Resultados/Impactos}

Resultados.

\section{Orçamento}

Orçamento, caso exista.
 \end{comment}

\section{Cronograma do Trabalho}

Segue abaixo o cronograma de trabalho das atividades realizadas e das que serão executadas até a Avaliação Final de TCC.

\begin{enumerate}
	\item Elaboração da Introdução
	\item Elaboração da Revisão da Literatura
	\item Elaboração da Metodologia
	\item Estudo sobre a \textit{Unity} e Iluminação
	\item Desenvolvimento da ferramenta
    \item Elaboração dos Resultados e Conclusões
\end{enumerate}

\FloatBarrier
\begin{figure*}[!htbp]
	\centering
	\includegraphics[scale=0.6]{imagens/Cronograma.png}
\end{figure*}
\FloatBarrier


